\documentclass[a4paper]{article}

\usepackage[utf8]{inputenc}
\usepackage[T1]{fontenc}
\usepackage{amsmath}
\usepackage{palatino} % times is boring (but more concise)
\usepackage{xcolor}
\usepackage{listings}
\usepackage{mathpartir}
\usepackage[margin=1.5in]{geometry}

\title{
  \color{blue}Fundamentals of Programming Languages\\[1ex]
  Assignment 4\\[1ex]
  Universal Types}
\author{Mestrado em (Engenharia) Informática\\
  Faculdade de Ciências da Universidade de Lisboa
}
\date{2022/2023}

\begin{document}
\maketitle
% \raggedright
\setlength{\parskip}{1ex}
\thispagestyle{empty}

\section{Binary Sums in System F}
\begin{enumerate}
  \item type Sum = $\forall$ T U V . (T -> V) -> (U -> V) -> V
  \item inl : $\forall$ T U . T -> ($\forall$ V . (T -> V) -> (U -> V) -> V) \\
        inl = \textbackslash T . \textbackslash U . \textbackslash l:T . \textbackslash V . \textbackslash t:(T -> V) . \textbackslash u:(U -> V) . t l
  \item inr : $\forall$ T U . U -> ($\forall$ V . (T -> V) -> (U -> V) -> V) \\
  inr = \textbackslash T . \textbackslash U . \textbackslash r:U . \textbackslash V . \textbackslash t:(T -> V) . \textbackslash u:(U -> V) . u r
  \item cases : $\forall$ T U V . ($\forall$ V . (T -> V) -> (U -> V) -> V) -> (T -> V) -> (U -> V) -> V \\
  cases = \textbackslash T . \textbackslash U . \textbackslash V . \textbackslash v:($\forall$ V . (T -> V) -> (U -> V) -> V) . \textbackslash lc:(T -> V) . \textbackslash rc:(U -> V) . v [V] lc rc
  \item cases [T] [U] [T] (inl [T] [U] v) (\textbackslash $x_1$ . $t_1$) (\textbackslash $x_2$ . $t_2$) = \\
    (\textbackslash V . \textbackslash t:(T -> V) . \textbackslash u:(U -> V) . t v) [T] (\textbackslash $x_1$ . $t_1$) (\textbackslash $x_2$ . $t_2$) = \\
    (\textbackslash t:(T -> T) . \textbackslash u:(U -> T) . t v) (\textbackslash $x_1$ . $t_1$) (\textbackslash $x_2$ . $t_2$) = \\
    (\textbackslash u:(U -> T) . (\textbackslash $x_1$ . $t_1$) v) (\textbackslash $x_2$ . $t_2$) = \\
    (\textbackslash $x_1$ . $t_1$) v \\\\
    This result is equivalent to [$x_1$ -> $v$]$t_1$, proving that it reduces to the latter result.\\
    We can notice that this looks similar to a more generic implementation of what we have in fromL (2.2).
  \item cases [T] [U] [U] (inr [T] [U] v) (\textbackslash $x_1$ . $t_1$) (\textbackslash $x_2$ . $t_2$) = \\
    (\textbackslash V . \textbackslash t:(T -> V) . \textbackslash u:(U -> V) . u v) [U] (\textbackslash $x_1$ . $t_1$) (\textbackslash $x_2$ . $t_2$) = \\
    (\textbackslash t:(T -> U) . \textbackslash u:(U -> U) . u v) (\textbackslash $x_1$ . $t_1$) (\textbackslash $x_2$ . $t_2$) = \\
    (\textbackslash u:(T -> U) . u v) (\textbackslash $x_2$ . $t_2$) = \\
    (\textbackslash $x_2$ . $t_2$) v \\\\
    This result is equivalent to [$x_2$ -> $v$]$t_2$, proving that it reduces to the latter result.\\
    We can notice that this looks similar to a more generic implementation of what we have in fromR (2.3).
\end{enumerate}

\end{document}

%%% Local Variables:
%%% mode: latex
%%% TeX-master: t
%%% End:
